% Include Preamble 
\documentclass[10pt,dvipsnames, aspectratio=169]{beamer}
\usetheme[progressbar=frametitle]{metropolis}
\usepackage[]{verbatim}\usepackage[]{}
\usepackage{booktabs}
\usepackage[scale=2]{ccicons}
\usepackage[misc]{ifsym}
\usepackage{wasysym}
\usepackage{listings}
\usepackage{xspace}
\usepackage{amsmath}
\usepackage{xcolor}
\usepackage{ragged2e}\justifying % for justify content % Customize 
\setlength{\parskip}{5pt} % vertical spacing between 2 paragraphs
\setbeamersize{text margin left=12mm, text margin right=12mm} 
\setbeamertemplate{frametitle}[default][left, leftskip=8mm]
%\usepackage[hidelink]{hyperref}
\newcommand{\themename}{\textbf{\textsc{metropolis}}\xspace}

% WARNING: Don't Touch,if you are not compfortable! 
\addtobeamertemplate{frametitle}{}{
%\begin{tikzpicture}[remember picture,overlay]
%\node[anchor=north east,yshift=2pt,xshift=-65pt] at (current page.north east) 
%{\includegraphics[height=.75cm]{img/elixir-portugal-white}};
%\node[anchor=north east,yshift=4pt] at (current page.north east) 
%{\includegraphics[height=1cm]{img/hdro}};
%\end{tikzpicture}
}

% Title Page of Presentation 
% WARNING: Don't change logo, just write your details 
\title{Health Research ToolBox: A Step by Step Guide for Beginners}
\date{\today}
\author{ Jubayer Hossain 
	https://jhossain.me/}
\institute{Lead Organizer, Introduction to Scientific Computing for Biologists 
\\ Founder, \\ Center for Health Innovation, Research, Action and Learning}
\titlegraphic{\vspace{4cm}\hfill\includegraphics[height=1cm]{img/jnulogo}}
\begin{document}

\maketitle
% Sectuion Title 
\section{Variable Distribution Type Tests (Gaussian)}

% Slide 1 
\begin{frame}[t]{Variable Distribution Type Tests}
	\begin{itemize}
		\item Shapiro-Wilk Test
		\item D’Agostino’s $k^{2}$ Test
		\item Anderson-Darling Test
	\end{itemize}
\end{frame}
% Slide 2 (Shapiro-Wilk Test)
\begin{frame}[t]{Shapiro-Wilk Test}
	Tests whether a data sample has a Gaussian distribution/normal distribution. \\
	\textbf{Assumptions}\\
		Observations in each sample are independent and identically distributed (iid).\\
	\textbf{Interpretation}	\\
	\begin{itemize}
		\item H0: The sample has a Gaussian/normal distribution.
		\item Ha: The sample does not have a Gaussian/normal distribution.
	\end{itemize}
\end{frame}
% Slide 3 (D’Agostino’s $k^{2}$ Test)
\begin{frame}[t]{D’Agostino’s $k^{2}$ Test}
	Tests whether a data sample has a Gaussian distribution/normal distribution. \\
	\textbf{Assumptions}\\
	Observations in each sample are independent and identically distributed (iid).\\
	\textbf{Interpretation}	\\
	\begin{itemize}
		\item H0: The sample has a Gaussian/normal distribution.
		\item Ha: The sample does not have a Gaussian/normal distribution.
	\end{itemize}

	\textbf{Remember} \\
	\begin{itemize}
	\item If Data Is Gaussian:
		\begin{itemize}
			\item Use Parametric Statistical Methods
		\end{itemize}
	\item Else:
	\begin{itemize}
		\item Use Nonparametric Statistical Methods
	\end{itemize}
	\end{itemize}
\end{frame}
% Sectuion Title 
\section{Variable Relationship Tests (correlation)}
% Slide 4 (Variable Relationship Tests (correlation))
\begin{frame}[t]{Variable Relationship Tests}
	\begin{itemize}
		\item Pearson’s Correlation 
		\item Coefficient
		\item Spearman’s Rank Correlation
		\item Kendall’s Rank Correlation
		\item Chi-Squared Test
	\end{itemize}
\end{frame}

% Slide 5 (Correlation Test)
\begin{frame}[t]{Correlation Test}
	Correlation Measures whether greater values of one variable correspond to greater values in the other. Scaled to always lie between $\pm$ 1 .
	\begin{itemize}
		\item Correlation is Positive when the values increase together.
		\item Correlation is Negative when one value decreases as the other increases.
		\item A correlation is assumed to be linear.
		\item 1 is a perfect positive correlation
		\item 0 is no correlation (the values don’t seem linked at all)
		\item --1 is a perfect negative correlation
	\end{itemize}
\end{frame}
% Slide 6 (Correlation Methods)
\begin{frame}[t]{Correlation Methods}
	\begin{itemize}
		\item \textbf{Pearson's Correlation Test:} \\ assumes the data is normally distributed and measures linear correlation.
		\item \textbf{Spearman's Correlation Test:} \\ does not assume normality and measures non-linear correlation.
		\item \textbf{Kendall's Correlation Test:} \\ similarly does not assume normality and measures non-linear correlation, but it less commonly used.
		
	\end{itemize}
\end{frame}
% Slide 7 (Difference)
\begin{frame}[t]{Difference Between Pearson's and Spearman's}
	\begin{tabular}{ l l }
		\textbf{Pearson's Test} & \textbf{Spearman's Test}  \\
		\hline
		Paramentric Correlation & Non-parametric \\
		Linear relationship & Non-linear relationship \\
		Continuous variables & continuous or ordinal variables \\
		Propotional change & Change not at constant rate
		
	\end{tabular}
\end{frame}

% Slide 8 (Pearson’s Correlation Coefficient)
\begin{frame}[t]{Pearson’s Correlation Coefficient}
	Tests whether two samples have a linear relationship. \\
	\textbf{Assumptions}
	\begin{itemize}
		\item Observations in each sample are independent and identically distributed (iid).
		\item Observations in each sample are normally distributed.
		\item Observations in each sample have the same variance.
	\end{itemize}

	\textbf{Interpretation}
\begin{itemize}
	\item H0: There is a relationship between two variables
	\item Ha: There is no relationship between two variables
\end{itemize}
\end{frame}

% Slide 9 (Spearman’s Rank Correlation Test)
\begin{frame}[t]{Spearman’s Rank Correlation Test}
	Tests whether two samples have a monotonic relationship. \\
	\textbf{Assumptions}
	\begin{itemize}
		\item Observations in each sample are independent and identically distributed (iid).
		\item Observations in each sample can be ranked.
	\end{itemize}
	
	\textbf{Interpretation}
	\begin{itemize}
		\item \textbf{H0 hypothesis:} \\ There is is relationship between variable 1 and variable 2
		\item \textbf{H1 hypothesis:} \\ There is no relationship between variable 1 and variable 2
	\end{itemize}
\end{frame}

% Slide 10 (Kendall’s Rank Correlation Test)
\begin{frame}[t]{Kendall’s Rank Correlation Test}
	%Tests whether two samples have a monotonic relationship. \\
	\textbf{Assumptions}
	\begin{itemize}
		\item Observations in each sample are independent and identically distributed (iid).
		\item Observations in each sample can be ranked.
	\end{itemize}
	
	\textbf{Interpretation}
	\begin{itemize}
		\item \textbf{H0 hypothesis:} \\ There is is relationship between variable 1 and variable 2
		\item \textbf{H1 hypothesis:} \\ There is no relationship between variable 1 and variable 2
	\end{itemize}
\end{frame}

% Slide 11 (Chi-Squared Test)
\begin{frame}[t]{Chi-Squared Test (Con..)}

	\begin{itemize}
		\item The Chi-square test of independence tests if there is a significant relationship between two categorical variables
		The test is comparing the observed observations to the expected observations.
		\item The data is usually displayed in a cross-tabulation format with each row representing a category for one variable and each column representing a category for another variable.
		\item Chi-square test of independence is an omnibus test. Meaning it tests the data as a whole. This means that one will not be able to tell which levels (categories) of the variables are responsible for the relationship if the Chi-square table is larger than 2×2
		\item If the test is larger than 2×2, it requires post hoc testing. If this doesn’t make much sense right now, don’t worry. Further explanation will be provided when we start working with the data.
	\end{itemize}
\end{frame}
%slide 12 (Chi-Squared Test (Con..))
\begin{frame}[t]{Chi-Squared Test (Con..)}
	\textbf{Assumptions} \\
		\begin{itemize}
		\item It should be two categorical  variables(e.g; Gender)
		\item Each variables should have at leats two groups(e.g; Gender = Female or Male)
		\item There should be independence of observations(between and within subjects)
		\item Large sample size
		\begin{itemize}
			\item The expected frequencies should be at least 1 for each cell.
			\item The expected frequencies for the majority(80\%) of the cells should be at least 5.
		\end{itemize}
	\end{itemize}
		If the sample size is small, we have to use \textbf{Fisher's Exact Test}
		
		\textbf{Fisher's Exact Test} is similar to Chi-squared test, but it is used for small-sized samples.
\end{frame}


%slide 12 (Chi-Squared Test )
\begin{frame}[t]{Chi-Squared Test}
\textbf{Interpretation}
\begin{itemize}
	\item The H0 (Null Hypothesis): There is a relationship between variable one and variable two.
	\item The Ha (Alternative Hypothesis): There is no relationship between variable 1 and variable 2.
\end{itemize}

\end{frame}

%slide 13 (Contingency Table)
\begin{frame}[t]{Contingency Table}
	Contingency table is a table with at least two rows and two columns(2x2) and its use to present categorical data in terms of frequency counts.	
\end{frame}















% Thank you slide 
\plain{Thank You\\ \ \\ \Huge{\smiley}}

\end{document}