% Include Preamble 
\documentclass[10pt,dvipsnames, aspectratio=169]{beamer}
\usetheme[progressbar=frametitle]{metropolis}
\usepackage[]{verbatim}\usepackage[]{}
\usepackage{booktabs}
\usepackage[scale=2]{ccicons}
\usepackage[misc]{ifsym}
\usepackage{wasysym}
\usepackage{listings}
\usepackage{xspace}
\usepackage{amsmath}
\usepackage{xcolor}
\usepackage{ragged2e}\justifying % for justify content % Customize 
\setlength{\parskip}{5pt} % vertical spacing between 2 paragraphs
\setbeamersize{text margin left=12mm, text margin right=12mm} 
\setbeamertemplate{frametitle}[default][left, leftskip=8mm]
%\usepackage[hidelink]{hyperref}
\newcommand{\themename}{\textbf{\textsc{metropolis}}\xspace}

% WARNING: Don't Touch,if you are not compfortable! 
\addtobeamertemplate{frametitle}{}{
%\begin{tikzpicture}[remember picture,overlay]
%\node[anchor=north east,yshift=2pt,xshift=-65pt] at (current page.north east) 
%{\includegraphics[height=.75cm]{img/elixir-portugal-white}};
%\node[anchor=north east,yshift=4pt] at (current page.north east) 
%{\includegraphics[height=1cm]{img/hdro}};
%\end{tikzpicture}
}

% Title Page of Presentation 
% WARNING: Don't change logo, just write your details 
\title{Health Research ToolBox: A Step by Step Guide for Beginners}
\date{\today}
\author{ Jubayer Hossain 
	https://jhossain.me/}
\institute{Lead Organizer, Introduction to Scientific Computing for Biologists 
\\ Founder, \\ Center for Health Innovation, Research, Action and Learning}
\titlegraphic{\vspace{4cm}\hfill\includegraphics[height=1cm]{img/jnulogo}}

\begin{document}
\maketitle
% Sectuion Title 
\section{Compare Sample Means (parametric)}
% Slide (Compare Sample Means)
\begin{frame}[t]{Compare Sample Means}
    \begin{itemize}
      \item Student’s t-test
      \item Paired Student’s t-test
      \item Analysis of Variance Test (ANOVA)
      \item Repeated Measures ANOVA Test
	\end{itemize}
\end{frame}
% Slide 2 (t-test)
\begin{frame}[t]{t-test}
	\begin{itemize}
		\item It compares \textbf{mean} of \textbf{two groups}
		\item It is a parametric statistical test.
		\item It's used to study if there is \textbf{statistical difference} between \textbf{two groups}
	\end{itemize}
\end{frame}

% Slide 3 (Types of t-test)
\begin{frame}[t]{Types of t-test}
	\begin{itemize}
		\item One sample t-test
		\item Paired t-test(Dependent)
		\item Unpaired t-test(Independent)
	\end{itemize}
	Unpaired t-test also have 2 categories \\
		\begin{itemize}
			\item Student's t-test
				\begin{itemize}
				\item Equal variance
				\item Two sample t-test
				\end{itemize}
			\item Welch t-test
				\begin{itemize}
				\item Unequal variance
				\item Unequal variance t-test
				\end{itemize}
		\end{itemize}
\end{frame}

% Slide 4 (t-test)
\begin{frame}[t]{Selection of t-test}
	\begin{itemize}
		\item One sample t-test(for one sample)
		\item Paired t-test(for dependent samples)
		\item Student t-test(When sample size and variance are equal)
		\item Welch t-test(When sample size and variance are different)
	\end{itemize}
\end{frame}
% Slide 5 (t-test)
\begin{frame}[t]{One Sample t-test}
	It compares the mean of one sample \\
	\begin{itemize}
		\item Known(from previous study) mean ( $\mu$ )
		\item Hypothetical mean( $\mu$ )
	\end{itemize}
\end{frame}

% Slide 6 (Student's t-test)
\begin{frame}[t]{Student's t-test (Con..)}
	\begin{itemize}
		\item The independent t-test is also called the two sample t-test, student’s t-test, or unpaired t-test.
		\item It’s an univariate test that tests for a significant difference between the mean of two unrelated groups.
		\item It compares the mean of two independent samples.
	\end{itemize}
	\textbf{Assumptions} \\
	The assumptions that the data must meet in order for the test results to be valid are: \\
		\begin{itemize}
		\item The independent variable (IV) is categorical with at least two levels (groups)
		\item The dependent variable (DV) is continuous which is measured on an interval or ratio scale
		\item The distribution of the two groups should follow the normal distribution
		\item The variances between the two groups are equal
		\item This can be tested using statistical tests including Levene’s test, F-test, and Bartlett’s test.
	\end{itemize}
\end{frame}
% Slide 7 (Student's t-test)
\begin{frame}[t]{Student's t-test (Con..)}
	If any of these assumptions are violated then another test should be used. \\
	\textbf{Interpretation}\\
	\textbf{Question: Is there a difference in the height between men and women? \\
		Hypothesis}
	\begin{itemize}
		\item H0: the means of the samples are equal.
		\item Ha: the means of the samples are unequal.
	\end{itemize}
	\textbf{References} \\
	\url{https://pythonfordatascienceorg.wordpress.com/independent-t-test-python/}
\end{frame}
% Slide 8 (The Hypothesis Being Tested)
\begin{frame}[t]{The Hypothesis Being Tested}
	\begin{itemize}
		\item Null Hypothesis (H0): u1 = u2, which translates to the mean of sample$\_$01 is equal to the mean of sample 02
		\item Alternative Hypothesis (H1): u1 $\neq$ u2, which translates to the means of sample01 is not equal to sample 02
	\end{itemize}
	\textbf{Homogeneity of variance} \\
	Of these tests, the most common assessment for homogeneity of variance is Levene's test. The Levene's test uses an F-test to test the null hypothesis that the variance is equal across groups. A p value less than .05 indicates a violation of the assumption. \\
	\url{https://en.wikipedia.org/wiki/Levene\%27s_test} \\
	\url{https://docs.scipy.org/doc/scipy-0.14.0/reference/generated/scipy.stats.levene.html}
\end{frame}
% Slide 9 (Levene's test)
\begin{frame}[t]{Levene's test}
	Levene's test is an inferential statistic used to assess the equality of variances for a variable calculated for two or more groups. \\
	\textbf{Interpretation}
	\begin{itemize}
		\item H0: The variances are equal between two groups
		\item Ha: The variances are not equal between two groups
	\end{itemize}
\end{frame}

% Slide 10 (Checking normal distribution)
\begin{frame}[t]{Checking normal distribution by shapiro method}
	\begin{itemize}
		\item \url{https://docs.scipy.org/doc/scipy/reference/generated/scipy.stats.shapiro.html}
		\item \url{https://stats.stackexchange.com/questions/15696/interpretation-of-shapiro-wilk-test}
	\end{itemize}
\end{frame}
% Slide 11 (Paired t-test)
\begin{frame}[t]{Paired t-test}
	It compares the mean between two related samples.(each subject is measured twice)
\end{frame}

% Slide 12 (The Hypothesis Being Tested)
\begin{frame}[t]{The Hypothesis Being Tested}
		\begin{itemize}
		\item Null Hypothesis (H0): u1 = u2, which translates to the mean of sample 01 is equal to the mean of sample 02
		\item Alternative hypothesis (Ha): u1 $\neq$ u2, which translates to the means of sample 01 is not equal to sample 02
		\end{itemize}
	\textbf{Assumption check} \\
	\begin{itemize}
		\item The samples are independently and randomly drawn
		\item The distribution of the residuals between the two groups should follow the normal distribution
		\item The variances between the two groups are equal
	\end{itemize}
\end{frame}
% Slide 13 (Welch's t-test)
\begin{frame}[t]{Welch's t-test (Con..)}
	\begin{itemize}
		\item It compares the mean of two independent samples.
		\item It assumes:
		\begin{itemize}
			\item Samples don't have equal variance
			\item Sample size is not equal.
		\end{itemize}
	\end{itemize}
Welch's t-test Assumptions Like every test, this inferential statistic test has assumptions. The assumptions that the data must meet in order for the test results to be valid are: \\
	\begin{itemize}
	\item The independent variable (IV) is categorical with at least two levels (groups)
	\item The dependent variable (DV) is continuous which is measured on an interval or ratio scale
	\item The distribution of the two groups should follow the normal distribution If any of these assumptions are violated then another test should be used.
\end{itemize}
\end{frame}
%slide 14 (Welch's t-test)
\begin{frame}[t]{Welch's t-test}
	\textbf{Interpretation} \\
	\begin{itemize}
		\item Null hypothesis (H0): u1 = u2, which translates to the mean of sample 1 is equal to the mean of sample 2
		\item Alternative hypothesis (HA): u1 $\neq$ u2, which translates to the mean of sample 1 is not equal to the mean of sample 2
	\end{itemize}
\end{frame}

% Sectuion Title 
\section{Analysis of Variance(ANOVA)}
% Slide 
\begin{frame}[t]{ANOVA - Analysis of Variance}
	\begin{itemize}
		\item Compares the means of 3(+) groups of data.
		\item Used to study if there is \textbf{statistical difference} between 3(+) group of data.
		\item Assumes the data are \textbf{normally distributed} and have \textbf{equal variances}
	\end{itemize}
	
\end{frame}
% Slide 2 (One-way ANOVA)
\begin{frame}[t]{One-way ANOVA}
	\begin{itemize}
		\item Compares the mean of 3(+) groups of data considering one independent variable or factor.
		\item Within each group there should be at least three observations.
	\end{itemize}
\end{frame}
% Slide 3 (Two-way ANOVA)
\begin{frame}[t]{Two-way ANOVA}
	\begin{itemize}
		\item Compares the means of 3(+) groups of data considering two independent variables or factors.
	\end{itemize}
	\textbf{Assumptions}\\
	\begin{itemize}
		\item Observations in each sample are independent and identically distributed (iid).
		\item Observations in each sample are normally distributed.
		\item Observations in each sample have the same variance.
	\end{itemize}
	\textbf{Interpretation}\\
	\begin{itemize}
		\item H0: the means of the samples are equal.
		\item Ha: one or more of the means of the samples are unequal.
	\end{itemize}
\end{frame}


% Thank you slide 
\plain{Thank You\\ \ \\ \Huge{\smiley}}

\end{document}